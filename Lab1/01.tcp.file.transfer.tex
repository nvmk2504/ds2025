\documentclass[a4paper,12pt]{article}

% Packages for formatting and better output
\usepackage[utf8]{inputenc}
\usepackage{amsmath}
\usepackage{graphicx}
\usepackage{hyperref}
\usepackage{listings}
\usepackage{color}

% Set up page margins
\usepackage[margin=1in]{geometry}

\title{File Transfer over TCP/IP using Sockets\\ \large Lab Report}
\author{Nguyen Viet Minh Khoi - 22BI13221}

\begin{document}

\maketitle

\section{Introduction}
File transfer protocols play a crucial role in contemporary networking. This lab aims to develop a basic file transfer system utilizing the Transmission Control Protocol (TCP). By leveraging Python's socket library, we created a client-server application in which the client transmits a file to the server via a TCP/IP connection.

\section{Objectives}
The objectives of this lab were:
\begin{itemize}
    \item Create a file transfer system utilizing TCP sockets.
    \item Grasp client-server communication through socket programming.
    \item Investigate how data is transmitted in segments across a network.

\end{itemize}

\section{Methodology}
In this lab, a client-server architecture was established for transferring files over TCP/IP. The methodology was as follows:

\begin{enumerate}
    \item The server was set up to await incoming connections from clients on a designated port.
    \item After a client connected, the server received the file from the client in segments, allowing for efficient handling of large files.
    \item The client established a connection to the server, transmitted the filename, and subsequently sent the file data in 1024-byte segments.
    \item The server stored the file on disk after it had received the complete file.
    \item Both the client and server terminated the connection after the transfer was finished.
\end{enumerate}

\section{System Design}
The system consists of two primary components: the server and the client. The server monitors a specified port, accepts incoming connections, and receives files from the client. Meanwhile, the client connects to the server, transmits the file, and closes the connection after the transfer is finished.

\subsection{File Transfer Protocol}
The file transfer protocol follows a simple sequence:
\begin{enumerate}
    \item The client transmits the filename to the server.
    \item The client subsequently sends the file in 1024-byte segments.
    \item The server receives each segment and writes it to a file on the disk.
    \item After the file transfer is finished, the server verifies the transfer, and both the client and server terminate the connection.
\end{enumerate}

\section{Implementation}
The file transfer system was implemented using Python's socket library. The following code illustrates the server and client components.

\subsection{Server Code}

\begin{lstlisting}[language=Python, caption=Server Code]
    import socket

    HOST = '127.0.0.1'  
    PORT = 25041   
    
    se = socket.socket(socket.AF_INET, socket.SOCK_STREAM)
    se.bind((HOST, PORT))
    
    se.listen()
    print(f"Server listening on {HOST}:{PORT}")
    
    con, addr = se.accept()
    print(f"Connected by {addr}")
    
    fn = con.recv(1024).decode('utf-8') 
    print(f"Receiving file: {fn}")
    
    
    with open(fn, 'wb') as f:
        while True:
            
            data = con.recv(1024)
            if not data:
                break  
            f.write(data)
    
    print(f"Successfully.")
    con.close() 
    se.close()
\end{lstlisting}

\subsection{Client Code}

\begin{lstlisting}[language=Python, caption=Client Code]
    import socket

    def cl():
        host = '127.0.0.1'
        port = 25041
        fn = 'received_transfer_file.txt'
    
        try:
            with socket.socket(socket.AF_INET, socket.SOCK_STREAM) as cs:
                cs.connect((host, port))
                print(f"Connected to server {host}:{port}")
    
                with open(fn, 'wb') as fw:
                    while (data := cs.recv(1024)):
                        fw.write(data)
    
                print(f"successfully")
        except Exception as e:
            print(f"Error")
    
    if __name__ == '__main__':
        cl()
\end{lstlisting}

\subsection{Sample Output from the Client}

\begin{verbatim}
Server listening on 127.0.1.125:2002...
Connected by ('127.0.0.1', 33653)
Receiving file: example.txt
File example.txt received successfully.
\end{verbatim}

The file `example.txt` was successfully transferred from the client to the server, and the server saved it in the current directory.

\section{Discussion}
This lab showcased the fundamental concepts of file transfer over a TCP/IP connection through socket programming. The system effectively transferred files by dividing them into smaller segments, allowing it to manage larger files efficiently.

Furthermore, the client-server system could be enhanced to support multiple file transfers concurrently.

\section{Conclusion}
This lab demonstrated how to utilize TCP/IP sockets for transferring files between a client and a server. By adhering to the essential principles of socket programming, we successfully created a file transfer system that efficiently sends and receives files. 

\end{document}
