\documentclass[12pt]{article}
\usepackage{amsmath, amssymb, amsthm} % For mathematical symbols and theorems
\usepackage{graphicx} % For including images
\usepackage{geometry} % For adjusting page margins
\geometry{margin=1in} % 1-inch margins

\begin{document}

% Title
\title{Sample \LaTeX\ File}
\author{NguyenVietMinhKhoi-22BI13221}
\maketitle

% Abstract
\begin{abstract}
This document represents the output from the file ``sample.tex'' once compiled using your favorite \TeX\ compiler. This file should serve as a good example of the basic structure of a ``.tex'' file as well as many of the most basic commands needed for typesetting documents involving mathematical symbols and expressions. For more of a description on how each command works, please consult the links found on our course webpage.
\end{abstract}

% Section 1: Lists
\section{Lists}
\begin{enumerate}
    \item \textbf{First Point (Bold Face)}
    \item \textit{Second Point (Italic)}
    \item \Huge Third Point (Large Font)
    \begin{enumerate}
        \item \small First Subpoint (Small Font)
        \item \tiny Second Subpoint (Tiny Font)
        \item \Huge Third Subpoint (Huge Font)
    \end{enumerate}
\end{enumerate}
\begin{itemize}
    \item \textsf{Bullet Point (Sans Serif)}
    \item \textsc{Circle Point (Small Caps)}
\end{itemize}

% Section 2: Equations
\section{Equations}

\subsection{Binomial Theorem}
\begin{theorem}[Binomial Theorem]
For any nonnegative integer $n$, we have
\[
(1+x)^n = \sum_{i=0}^n \binom{n}{i} x^i
\]
\end{theorem}

\subsection{Taylor Series}
The Taylor series expansion for the function $e^x$ is given by
\[
e^x = 1 + x + \frac{x^2}{2} + \frac{x^3}{6} + \cdots = \sum_{n=0}^\infty \frac{x^n}{n!} \tag{1}
\]

\subsection{Sets}
\begin{theorem}
For any sets $A$, $B$, and $C$, we have
\[
(A \cup B) - (C - A) = A \cup (B - C)
\]
\end{theorem}

\noindent \textbf{Proof:}
\[
\begin{aligned}
(A \cup B) - (C - A) &= (A \cup B) \cap (C - A)^c \\
&= (A \cup B) \cap (C^c \cup A) \\
&= (A \cup B \cap C^c) \cup (A \cap A) \\
&= A \cup (B \cap C^c) \\
&= A \cup (B - C)
\end{aligned}
\]

% Section 3: Tables
\section{Tables}
\begin{center}
\begin{tabular}{|l|c|r|}
\hline
left justified & center & right justified \\ \hline
1 & 3.14159 & 5 \\ 
2.4678 & 3 & 1234 \\ 
3.4678 & 6.14159 & 1239 \\ \hline
\end{tabular}
\end{center}

% Section 4: A Picture
\section{A Picture}
\setlength{\unitlength}{0.5cm}
\begin{picture}(10,5)
    \thicklines
    % Eyebrows
    \put(1,4){\line(1,0){2}}
    \put(7,4){\line(1,0){2}}
    % Eyes
    \put(2,2.5){\circle{1}}
    \put(8,2.5){\circle{1}}
    % Nose
    \put(5,2){\line(1,2){0.5}}
    \put(5.5,3){\line(1,-2){0.5}}
    \put(5,2){\line(1,0){1}}
    % Mouth
    \put(3,1){\line(1,0){4}}
    \put(3,1){\line(0,-1){1}}
    \put(7,1){\line(0,-1){1}}
    \put(3,0){\line(1,0){4}}
\end{picture}

\end{document}
