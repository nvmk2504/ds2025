\documentclass[12pt]{article}
\usepackage{graphicx}
\usepackage{geometry}
\geometry{a4paper, margin=1in}

\title{Practical work 2}

\begin{document}
\author{Nguyen Viet Minh Khoi}
\maketitle

\section*{RPC Service Design}
The RPC service is built using Python's built-in xmlrpc library. The server provides a function called send_file(), which opens a file in binary mode and transfers its contents to the client.


\section*{System Organization}
Two components are:
\begin{enumerate}
    \item \textbf{Server:} Hosts the RPC function for file transfer.
    \item \textbf{Client:} Requests the file and saves it locally.
\end{enumerate}


\section*{Implementation}
\subsection*{Code Snippet: Server}
\begin{verbatim}
    from xmlrpc.server import SimpleXMLRPCServer

    def save(name, data):
    
        try:
            with open(name, 'wb') as file:
                file.write(data.data)
            return f"File '{name}' received and saved."
        except Exception as e:
            return f"Failed to save file: {e}"
    
    if __name__ == '__main__':
    
        server = SimpleXMLRPCServer(('127.0.0.1', 25041), allow_none=True)
        server.register_function(save, "save_file")
        print("Server running at 127.0.0.1:25041")
        server.serve_forever()
\end{verbatim}

\subsection*{Code Snippet: Client}
\begin{verbatim}
    import xmlrpc.client

    def upload(name):
    
        try:
            with open(name, 'rb') as file:
                binary_data = file.read()
            response = rpc_server.save_file(name, xmlrpc.client.Binary(binary_data))
            print(response)
        except Exception as e:
            print(f"Error: {e}")
    
    if __name__ == '__main__':
    
        rpc_server = xmlrpc.client.ServerProxy('http://127.0.0.1:25041')
    
        file_to_send = 'transfer_file.txt'
        upload(file_to_send)
\end{verbatim}

\end{document}